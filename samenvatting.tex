% ************************************************************************
%
% Samenvatting
%
% ************************************************************************
\noquote
\selectlanguage{dutch}

\chpos{22mm}{12mm}
\chapter*{Samenvatting}
\markboth{Samenvatting}{Samenvatting}
\addcontentsline{toc}{chapter}{Samenvatting}

\lettrine{N}{euronen} behoren tot de belangrijkste elementen van het zenuwstelsel. Fascinatie voor deze cellen gaat minstens terug tot het baanbrekende werk van Ram\'{o}n y Cajal, nu meer dan een eeuw geleden. Gewapend met een microscoop en gebruikmakend van zilverkleuring, niet lang daarvoor ontdekt door Golgi, bestudeerde hij hersenweefsels uit een groot aantal gebieden van de hersenen. Zijn bevindingen leidden tot het opstellen van de neuronentheorie, die stelt dat het zenuwstelsel, net als alle andere organen in het lichaam, is opgebouwd uit afzonderlijke cellen. Golgi en Ram\'{o}n y Cajal deelden in 1906 de Nobelprijs voor Geneeskunde. Daarop volgend grondig onderzoek naar neuronen onthulde dat deze cellen de bijzondere eigenschap hebben dat ze signalen kunnen ontvangen en doorgeven. Daarmee regelen ze een groot aantal lichaamsfuncties. Ook werd duidelijk dat neuronen in de verschillende delen van de hersenen verschillende functies hebben. Afhankelijk van hun specifieke rol in het zenuwstelsel kunnen neuronen nogal variëren in hun morfologische eigenschappen.

Morfologische analyse van de verschillende soorten neuronen is daarom vaak een belangrijk onderdeel in het onderzoek naar hun functie. Neuronen kunnen tegenwoordig in groot detail en digitaal worden afgebeeld door middel van moderne lichtmicroscopen. Maar om de eigenschappen van een gegeven neuron daadwerkelijk te kunnen kwantificeren is een explicietere representatie van zijn morfologie nodig dan een microscoopbeeld. Het afleiden van een grafische representatie van een neuron uit zijn microscoopbeeld, in de vorm van een boomstructuur bestaande uit knooppunten en vertakkingen, wordt doorgaans aangeduid als digitale reconstructie, en is het hoofdthema van dit proefschrift. Veel neurowetenschappelijke studies zijn afhankelijk van een nauwkeurige beschrijving van de morfologie van neuronen in de vorm van digitale reconstructies. Daarmee is digitale reconstructie een belangrijk technisch probleem in neurowetenschappelijk onderzoek.

Dit proefschrift presenteert nieuwe computationele methoden voor de automatische analyse van neuronen. De hoofdproblemen waarvoor oplossingen worden gepresenteerd zijn de detectie en de reconstructie van neuronen in digitale fluorescentiemicroscopiebeelden. Een van de belangrijkste vernieuwingen die worden voorgesteld ten opzichte van bestaande reconstructiemethoden is het gebruik van probabilistische filtertechnieken. Na een algemene inleiding in het eerste hoofdstuk, beschrijven de daarop volgende hoofdstukken originele oplossingen voor de automatische detectie van knooppunten en eindpunten van neuronen in hun afbeeldingen, het traceren van alle vertakkingen van de neuronen in de beelden, en het vinden van neuronen in lageresolutiebeelden uit screeningstudies. De rest van deze samenvatting geeft kort de inhoud van de hoofdstukken weer.

Het tweede hoofdstuk presenteert een nieuwe methode voor de automatische detectie van die punten in beelden van neuronen die van cruciaal belang zijn voor de correcte topologische representatie van de neuronen. Het gaat hierbij vooral om de knooppunten en de eindpunten van alle vertakkingen. Een knooppunt is een punt waar drie segmenten van de boomstructuur bij elkaar komen, en een eindpunt is een punt waar een vertakking van de boomstructuur eindigt. De voorgestelde detectiemethode maakt gebruik van richtingsfilters, waarmee in elk punt van een beeld wordt bepaald in hoeverre en in welke richting(en) er lijnachtige structuren door het punt lopen. De gevonden informatie hierover wordt uitgedrukt in taalkundige termen die vervolgens verwerkt worden door middel van zogeheten vage logica. Daarmee wordt elk punt met behulp van een stelsel van regels geclassificeerd als zijnde irrelevant of een specifiek type cruciaal punt uit de boomstructuur van het neuron. Voor dit doel wordt een nieuw stelsel van regels en klassen voorgesteld. De kracht van de gekozen aanpak is dat voor elk punt berekend wordt in welke mate het behoort tot elk van de beschouwde klassen. Daardoor wordt rekening gehouden met de onzekerheid in de beeldinformatie en het filterproces.

In het derde hoofdstuk wordt het theorema van Bayes benut om te komen tot een nieuwe automatische methode voor het traceren van de middellijn van neuronale vertakkingen in microscoopbeelden. In tegenstelling tot bestaande methoden is de voorgestelde methode probabilistisch van aard en in staat om tegelijkertijd een onbeperkt aantal vertakkingen te traceren. Voor de implementatie van de methode wordt gebruik gemaakt van sequentiële Monte Carlo filtering. Een dergelijke aanpak wordt ook wel gebruikt in andere toepassingen, voor het volgen van bewegende objecten over de tijd in filmopnames. In dit hoofdstuk wordt het idee echter aangewend voor het volgen van objecten in de ruimte in statische opnames. Om dit mogelijk te maken worden nieuwe wiskundige modellen voorgesteld voor het filterproces, waarin bestaande kennis is opgenomen over de vorm van neuronale vertakkingen en de manier waarop ze worden afgebeeld door een microscoop. De probabilistische aard van de methode maakt dat herhaalde toepassing op hetzelfde beeld net iets andere resultaten oplevert. Op deze manier kan meer statistisch bewijsmateriaal worden vergaard over de vorm van de vertakkingen dan dat deterministische methoden kunnen leveren. De gepresenteerde experimentele resultaten bevestigen inderdaad dat de methode nauwkeuriger is.

Het vierde hoofdstuk tilt het idee van probabilistische tracing nog een stap verder en presenteert een nieuwe methode voor automatische volledige reconstructie van neuronen uit microscoopbeelden. Deze methode vindt niet alleen de middellijn van individuele vertakkingen, maar maakt ook een schatting van de lokale diameter op elk punt van de vertakkingen, en voegt alle gevonden segmenten samen tot een datastructuur die de berekening van allerlei morfologische eigenschappen mogelijk maakt. De methode begint met het identificeren van die gebieden in een beeld die hoogstwaarschijnlijk neuronale vertakkingen bevatten. Voor dit doel wordt een bestaande buisfiltermethode gebruikt. Uit de gevonden gebieden worden vervolgens startpunten geselecteerd voor het traceren van de vertakkingen. Net als in het voorgaande hoofdstuk wordt hiervoor een probabilistische aanpak gebruikt op basis van sequentiële Monte Carlo filtering. Ook hier levert herhaalde toepassing meer informatie over de vertakkingen en leidt tot betere resultaten. Tevens zij opgemerkt dat de herhalingen onafhankelijk zijn van elkaar en zich dus uitstekend lenen voor parallelle implementatie. Om een volledige reconstructie te verkrijgen worden de resultaten van de verschillende herhalingen verfijnd en samengevoegd. Hiervoor worden nieuwe iteratieve algoritmes voorgesteld. Een eerste versie van de methode is opgenomen in een internationale vergelijkingsstudie genaamd BigNeuron, waar het als een van de betere methoden uit de bus kwam. In dit hoofdstuk wordt een sterk verbeterde versie gepresenteerd.

Tenslotte wordt in het vijfde hoofdstuk een haalbaarheidsstudie gepresenteerd van het detecteren van gebieden in lageresolutiebeelden van celculturen die neuronen bevatten. Deze taak is doorgaans de eerste stap in screeningstudies naar de aantasting van neuronen door neurodegeneratieve ziektes en het effect van medicijnen. De gevonden gebieden worden vervolgens afgebeeld op hoge resolutie, waarna de neuronen kunnen worden gereconstrueerd met behulp van de in de vorige hoofdstukken beschreven methoden. De detectie in de lageresolutiebeelden wordt bemoeilijkt door de afwezigheid van details, het feit dat de neuronen vaak niet volledig zijn afgebeeld, de aanwezigheid van vergelijkbare cellen zoals astrocyten, en beeldvormingsartefacten zoals (veel) ruis. Daarom is in deze studie gekozen voor het gebruik van zelflerende methoden op basis van beeldkenmerken berekend door een zeer groot aantal bestaande filtertechnieken. In de gepresenteerde experimenten worden de prestaties van vier soorten traditionele zelflerende methoden vergeleken. Ook wordt een proefexperiment beschreven met een tegenwoordig zeer populaire aanpak op basis van kunstmatige, dieplerende neurale netwerken. De conclusie is echter dat op deze beperkte data de traditionele methoden beter presteren.

% ************************************************************************
\selectlanguage{english}
% ************************************************************************

