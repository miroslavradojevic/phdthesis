% ************************************************************************
%
% Automated Neuron Reconstruction from 3D Fluorescence Microscopy Images 
% using Sequential Monte Carlo Estimation
%
% ************************************************************************
\chpos{15mm}{8mm}
\chapter[Automated Neuron Reconstruction from 3D Fluorescence Microscopy Images using Sequential Monte Carlo Estimation]{Automated Neuron Reconstruction from 3D Fluorescence Microscopy Images using Sequential Monte Carlo Estimation}
\chaptermark{Automated Neuron Reconstruction using Sequential Monte Carlo Estimation}% from 3D Fluorescence Microscopy Images 
\label{ch4:pnr}

\myabstract{\lettrine{M}{icroscopic}  images of neuronal cells provide essential structural information about the key constituents of the brain and form the basis of many neuroscientific studies. Computational analyses of the morphological properties of the captured neurons require first converting the struc- tural information into digital tree-like reconstructions. Many dedicated computational methods and corresponding soft- ware tools have been and are continuously being developed with the aim to automate this step while achieving human- comparable reconstruction accuracy. This pursuit is ham- pered by the immense diversity and intricacy of neuronal morphologies as well as the often low quality and ambigu- ity of the images. Here we present a novel method we de- veloped in an effort to improve the robustness of digital re- construction against these complicating factors. The method is based on probabilistic filtering by sequential Monte Carlo estimation and uses prediction and update models designed specifically for tracing neuronal branches in microscopic im- age stacks. Moreover, it uses multiple probabilistic traces to arrive at a more robust, ensemble reconstruction. The pro- posed method was evaluated on fluorescence microscopy image stacks of single neurons and dense neuronal networks with expert manual annotations serving as the gold standard, as well as on synthetic images with known ground truth. The results indicate that our method performs well under varying experimental conditions and compares favorably to state-of- the-art alternative methods.}

%\smallskip
%\bigskip
%\vfill
\vspace{9em}
% ************************************************************************
\begin{publish}
	Based upon: M. Radojevi\'{c}, E. Meijering, ``Automated Neuron Reconstruction from 3D Fluorescence Microscopy Images using Sequential Monte Carlo Estimation'', \textit{Neuroinformatics}, vol. 0, no. 0, pp.0-0, 2018.   
\end{publish}

\section{Introduction}
\label{ch4_sec_intro}