% ************************************************************************
%
% Curriculum Vitae
%
% ************************************************************************

\noquote
\orgchpos
\chapter*{Curriculum Vitae}
\markboth{Curriculum Vitae}{Curriculum Vitae}
\addcontentsline{toc}{chapter}{Curriculum Vitae}

% ************************************************************************
\noindent
%Miroslav Radojevi\'{c} was born in U\v{z}ice, Serbia, on January 7, 1984. He received a B.Sc. in electrical engineering (signal processing and control systems) from the Faculty of Electrical Engineering, University of Belgrade, Serbia. Subsequently, he completed M.Sc. degree in Computer Vision and Robotics from the ViBot Erasmus Mundus programme co-organized by Heriot-Watt University (United Kingdom), Universit\`{e} de Bourgogne (France) and Universit\"{a}t de Girona (Spain) in 2011. 
%%From 2012 to 2016, he was a conducted the  Scientist at the Electrical Engineering department of the same university. During that period he carried out research in the field of nonlinear and chaotic dynamical systems. 
%\bigskip
%\noindent
%During , he was a Research Assistant (postmaster program ``Mathematics for Industry'') at the department of Mathematics and Computer Science of Technical University of Eindhoven, the Netherlands. In 2005 he graduated on the project ``Design and implementation of a six camera scanning unit'' and was awarded a Professional Doctorate in Engineering degree (PDEng). 
%\bigskip
%\noindent
%From Dec. 2012 to Feb. 2016 he was a Ph.D. student at the Departments of Medical Informatics and Radiology of the Erasmus University Rotterdam, the Netherlands. His research topic was tracking and motion analysis in cellular and molecular bioimaging. The project was carried out in collaboration with the Department of Cell Biology and Department of Pathology at Erasmus MC Rotterdam. The results are described in this thesis.
%\bigskip
%\noindent
%Since 2017. he has been working as software engineer with the research and development department of Becton, Dickinson and Company (BD) focused on the development of medical devices.