% ************************************************************************
%
% Automated neuron detection in high-content fluorescence microscopy images
% using machine learning 
%
% ************************************************************************
\chpos{15mm}{8mm}
\chapter[Automated neuron detection in high-content fluorescence microscopy images using machine learning]{Automated neuron detection in high-content fluorescence microscopy images using machine learning}
\chaptermark{Automated neuron detection in high-content images using machine learning}%in high-content fluorescence microscopy images 
\label{ch5:ndetchml}
% abstract
{\small \lettrine{T}{he} study of neuronal morphology in relation to function, and the development of effective medicines to positively impact this relationship in patients suffering from neurodegenerative diseases, increasingly involves image-based high-content screening and analysis. The first critical step toward fully automated high-content image analyses in such studies is to detect all neuronal cells and distinguish them from possible non-neuronal cells or artifacts in the images. Here we investigate the performance of well-established machine learning techniques for this purpose. These include
	support vector machines, random forests, k-nearest neighbors,
	and generalized linear model classifiers, operating on
	an extensive set of image features extracted using the compound
	hierarchy of algorithms representing morphology, and
	the scale-invariant feature transform. We present experiments
	on a dataset of rat hippocampal neurons from our own studies
	to find the most suitable classifier(s) and subset(s) of features
	in the common practical setting where there is very
	limited annotated data for training. The results indicate that
	a random forests classifier using the right feature subset ranks
	best for the considered task, although its performance is not
	statistically significantly better than some support vector machine
	based classification models.\par}
\vspace*{12em}
% ************************************************************************
\begin{publish}
	Based upon: G. Mata, M. Radojevi\'{c}, E. Meijering, ``Automated neuron detection in high-content fluorescence microscopy images using machine learning'', \textit{Neuroinformatics}, \textit{in review}
\end{publish}%vol. 0, no. 0, pp.0-0, 2018.
\section{Introduction}
\label{sec:intro}
