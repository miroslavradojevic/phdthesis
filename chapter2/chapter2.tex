% ************************************************************************
% 
%
% ************************************************************************

\myquote{0.55\textwidth}{Let no one say that I have said nothing
  new.\newline The arrangement of the subject is new.}{Blaise Pascal,
  \emph{Pens\'ees} (\oldstylenums{1670})\nocite{Pascal-1670a}}


\noquote

\chpos{15mm}{8mm}
\chapter[Quantitative Comparison of Spot Detection Methods in\\
 Fluorescence Microscopy]{Quantitative Comparison \\of Spot Detection Methods \\in Fluorescence Microscopy}
\chaptermark{Quantitative Comparison of Spot Detection Methods}
\label{ch2__chaprter2}

\mysquote{\textwidth}
{Not everything that can be counted counts, and not everything that
  counts can be counted.}
{Albert Einstein (1879-1955)}



% ************************************************************************
% ************************************************************************
\myabstract{
Quantitative analysis of biological image data generally involves the detection of many subresolution spots.
Especially in live cell imaging, for which fluorescence microscopy is often used, the signal-to-noise ratio (SNR) can be extremely low,
making automated spot detection a very challenging task. In the past, many methods have been proposed to perform this task,
but a thorough quantitative evaluation and comparison of these methods is lacking in the literature.
In this chapter, we evaluate the performance of the most frequently used detection methods for this purpose.
These include six unsupervised and two supervised methods.
We perform experiments on synthetic images of three different types,
for which the ground truth was available,
as well as on real image data sets acquired for two different biological studies,
for which we obtained expert manual annotations to compare with.
The results from both types of experiments suggest that for very low SNRs ($\approx$2),
the supervised (machine learning) methods perform best overall.
Of the unsupervised methods, the detector based on the so-called
$h$-dome transform from mathematical morphology 
performs comparably, and has the advantage that it does not require a cumbersome learning stage.
At high SNRs ($>$5), the difference in performance of all considered detectors becomes negligible.}

\smallskip

% ************************************************************************
% ************************************************************************

\begin{publish}
Based upon: M. Radojevi\'{c}, I. Smal, E. Meijering, ``Fuzzy logic based'', \textit{in press}.   
\end{publish}

\section{Introduction}
\label{ch2__sec:intro}
\dropping{3}{T}he very first stage in the analysis of biological image data generally deals with the detection of objects of interest.