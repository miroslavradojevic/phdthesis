% ************************************************************************
% 
% Fuzzy-Logic Based Detection and Characterization of Junctions and Terminations in Fluorescence Microscopy Images of Neurons
%
% ************************************************************************
\chpos{15mm}{8mm}
\chapter[Fuzzy-Logic Based Detection and Characterization of Junctions and Terminations in Fluorescence Microscopy Images of Neurons]{Fuzzy-Logic Based Detection and Characterization of Junctions and Terminations in Fluorescence Microscopy Images of Neurons}
\chaptermark{Fuzzy-Logic Based Detection of Junctions and Terminations}
\label{ch2:fuzzy}

% ************************************************************************
% ************************************************************************
% [lraise=0.1, nindent=0em, slope=-.5em]
\myabstract{\lettrine{D}{igital} reconstruction of neuronal cell morphology is an important step toward understanding the functionality
	of neuronal networks. Neurons are tree-like structures
	whose description depends critically on the junctions and
	terminations, collectively called critical points, making the
	correct localization and identification of these points a crucial
	task in the reconstruction process. Here we present
	a fully automatic method for the integrated detection and
	characterization of both types of critical points in fluorescence
	microscopy images of neurons. In view of the
	majority of our current studies, which are based on cultured
	neurons, we describe and evaluate the method for application
	to two-dimensional (2D) images. The method relies on
	directional filtering and angular profile analysis to extract
	essential features about the main streamlines at any location
	in an image, and employs fuzzy logic with carefully
	designed rules to reason about the feature values in order to
	make well-informed decisions about the presence of a critical
	point and its type. Experiments on simulated as well
	as real images of neurons demonstrate the detection performance
	of our method. A comparison with the output
	of two existing neuron reconstruction methods reveals that
	our method achieves substantially higher detection rates and
	could provide beneficial information to the reconstruction
	process.}

%\smallskip
%\bigskip
%\vfill
\vspace{10em}
% ************************************************************************
\begin{publish}
Based upon: M. Radojevi\'{c}, I. Smal, E. Meijering, ``Fuzzy-Logic Based Detection and Characterization of Junctions and Terminations in Fluorescence Microscopy Images of Neurons'', \textit{Neuroinformatics}, vol. 11, no. 11, pp.1-11, 2015.   
\end{publish}

\section{Introduction}
\label{ch2:sec:intro}%\dropping{3}{T}
The very first stage in the analysis of biological image data generally deals with the detection of objects of interest.