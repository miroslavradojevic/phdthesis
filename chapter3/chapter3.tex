% ************************************************************************
%
% Automated neuron tracing using probability hypothesis density filtering
%
% ************************************************************************
\chpos{15mm}{8mm}
\chapter[Automated neuron tracing using probability hypothesis density filtering]{Automated neuron tracing using probability hypothesis density filtering}
\chaptermark{Automated neuron tracing using probability hypothesis density filtering}
\label{ch3:phd}

\myabstract{\lettrine{T}{he} functionality of neurons and their role in neuronal networks is tightly connected to the cell morphology. A fundamental problem in many neurobiological studies aiming to unravel this connection is the digital reconstruction of neuronal cell morphology from microscopic image data. Many methods have been developed for this, but they are far from perfect, and better methods are needed. Here we present a new method for tracing neuron centerlines needed for full reconstruction. The method uses a fundamentally different approach than previous methods by considering neuron tracing as a Bayesian multi-object tracking problem. The problem is solved using probability hypothesis density filtering. Results of experiments on 2D and 3D fluorescence microscopy image datasets of real neurons indicate the proposed method performs comparably or even better than the state of the art.}
\vspace*{23em}
% ************************************************************************
\begin{publish}
	Based upon: M. Radojevi\'{c}, E. Meijering, ``Automated neuron tracing using probability hypothesis density filtering'', \textit{Bioinformatics}, vol. 33, no. 7, pp.1073-1080, 2017.   
\end{publish}

\section{Introduction}
\label{ch3_sec_intro}
Accurate reconstruction of the tree-like structure of neuronal cells from optical microscopy images is a crucial step in automating the analysis of single neuron morphology or the connectivity of neuronal networks \cite{meijering2010neuron, donohue2011automated, Peng-2015}. Microscopic images provide detailed information about the geometrical and topological properties of the neuronal arbors. Extracting and representing this information in a faithful and convenient digital format is key to many studies \cite{ascoli2002computational, ascoli2007neuromorpho, svoboda2011past, senft2011brief,halavi2012digital, Lu-2015}, as digital reconstructions enable neurobiologists to use computational approaches in addressing open issues in brain research, such as the relation between neuron structure and function, and the effects of neurodegenerative disease processes and drug compounds on neuron development and connectivity.